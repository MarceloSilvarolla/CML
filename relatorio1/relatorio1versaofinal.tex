\documentclass[12pt]{article}
%\usepackage[margin=1in]{geometry}% Change the margins here if you wish.
%\setlength{\parindent}{0pt} % This is the set the indent length for new paragraphs, change if you want.
%\setlength{\parskip}{5pt} % This sets the distance between paragraphs, which will be used anytime you have a blank line in your LaTeX code.
%\pagenumbering{gobble}% This means the page will not be numbered. You can comment it out if you like page numbers.

%------------------------------------
\usepackage{xcolor}
\usepackage{fancyvrb}
\usepackage[utf8]{inputenc}
\usepackage{hyperref}

% These packages allow the most of the common "mathly things"
\usepackage{amsmath,amsthm,amssymb}

% This package allows you to add images.
\usepackage{graphicx}
\usepackage{float}

\newcommand{\blu}{\textcolor{blue}}
\newcommand{\red}{\textcolor{red}}

% Should you need any additional packages, you can load them here. If you've looked up something (like on DeTeXify), it should specify if you need a special package.  Just copy and paste what is below, and put the package name in the { }.  
\usepackage{wasysym} %this lets me make smiley faces :-)

\title{CML - Uma Linguagem para {\it Machine Learning} \\ \Large Relatório 1 \blu{versão final} - Trabalho de Conceitos de Linguagem de Programação}

\author{Caio Lopes, Leonardo Blanger, Marcelo Silvarolla}

\date{11 de abril de 2018}

\begin{document}
\maketitle
\section{Prefácio}
\blu{Este é o relatório das fases 1 e 2 modificado de acordo com as alterações da linguagem feitas nas fases 3, 4 e 5. O que foi acrescentado está em azul. O que foi removido está em vermelho.}

\section{Requisitos do Domínio}

A linguagem a ser implementada tem como objetivo facilitar a utilização de técnicas tradicionais de Aprendizado de Máquina. Este domínio de aplicação vem recebendo crescente visibilidade nos últimos anos devido a diversos fatores. Podemos citar como principais motivos o aumento na produção de dados digitais, devido ao uso cada vez maior de soluções computacionais para diversas tarefas, tanto pessoais, como científicas e da indústria, além do surgimento e aprimoramento de algoritmos capazes de extrair informações úteis a partir destas grandes massas de dados, e o aumento da capacidade computacional, que permite a utilização destes algoritmos em uma escala prática.

Atualmente, projetos de software que realizem Aprendizado de Máquina geralmente utilizam linguagens de propósito geral, como {\it Python}, acrescidas de bibliotecas que agilizem tarefas específicas relacionadas a manipulação e processamento de dados.

Nesta perspectiva, este trabalho tem como objetivo o projeto e implementação de uma linguagem com funcionalidades próprias para as tarefas mais fundamentais de Aprendizado de Máquina. Esta linguagem deve permitir a manipulação de dados e fornecer comandos diretos para o treinamento e uso de algoritmos tradicionais de aprendizado.

De forma mais específica, a linguagem deverá prover comandos para leitura, gravação e manipulação de dados, de forma a criar uma interface amigável para o uso de {\it datasets} armazenados em planilhas no formato {\tt csv}. A linguagem deve permitir também um tratamento individual aos dados destes {\it datasets}, como por exemplo: separação de exemplos dentre conjunto de treinamento e de testes, separação dos atributos entre {\it features} e {\it labels}, conversão entre atributos categóricos e numéricos e normalização de valores.\footnote{\blu{As conversões e normalizações serão feitas de modo implícito pelos próprios algoritmos de aprendizado.}}

Além da manipulação de {\it datasets}, a linguagem deve prover também funções predefinidas associadas a implementações dos algoritmos mais tradicionais da área de Aprendizado de Máquina, tais como {\it Perceptron}, {\it Pocket-Perceptron}, Regressão Linear, Regressão Logística e Árvores de Decisão. Com estas funcionalidades, a linguagem busca tornar mais ágil e amigável o treinamento de modelos preditivos utilizando {\it datasets} do usuário.

Além do treinamento, a linguagem deve permitir o armazenamento e carregamento destes modelos previamente treinados, através de arquivos contendo informações sobre a estrutura dos modelos e os parâmetros resultantes do treinamento, para realização de inferência em dados inéditos de forma eficiente.

\subsection{Justificativas}

Dentre as motivações para o projeto desta linguagem, podemos citar a possibilidade de treinamento e utilização de algoritmos de aprendizado sem a necessidade de grande conhecimento técnico, devido ao encapsulamento do funcionamento interno destes algoritmos através das funções predefinidas da linguagem. Além disso, podemos citar também a busca pela facilidade e maior rapidez de implementação, além do risco menor de {\it bugs}. 

Com estas vantagens, seria possível a realização mais ágil de testes rápidos e esboços de sistemas de predição baseados nestes algoritmos, antes de uma implementação definitiva em linguagens e bibliotecas mais eficientes.

\subsection{Exemplo}

Abaixo é mostrado um esboço da utilização da linguagem, contendo os tipos {\tt dataset} e {\tt model}, que representam o conjunto de dados e o modelo preditivo a ser treinado sobre estes dados, respectivamente. Também são mostradas funções predefinidas para leitura e gravação de {\it datasets}, manipulação dos atributos destes {\it datasets}, treinamento de um modelo preditivo do tipo {\it Regressão Logística} e sua utilização para inferência em outros dados de entrada, além de leitura e gravação dos parâmetros deste modelo. Este exemplo corresponde ao treinamento de um modelo para predição de probabilidade de fraude em transações de cartões de crédito, utilizando um {\it dataset} público de transações\footnote{{\it Dataset} disponível em \url{https://www.kaggle.com/mlg-ulb/creditcardfraud}}.

\begin{verbatim}
dataset D = read_data(``creditcard.csv'');
D = categorical_to_binary(D, ``Class'', {``0'', ``1''})
dataset X = remove_columns(D, ``Class'');
dataset y = columns(D, ``Class'');
model M = logistic_regression(X, y, 10000);
...
dataset predictions = predict(M, x_test);
save_data(predictions, ``predictions.csv'');
...
save_model(M, ``model_name'');
M = read_model(``model_name'');
\end{verbatim}

\blu{A sintaxe final da linguagem possui algumas divergências em relação à que foi apresentada no exemplo anterior. A função \texttt{read\_data} se transformou em \texttt{load\_data}, e recebe como parâmetros o nome do aquivo e o caractere separador do \texttt{csv}. Além disso, a função \texttt{categorical\_to\_binary} não existe na linguagem, pois a tarefa de converter atributos categóricos para numéricos e binários é realizada internamente no treinamento dos modelos e predição de dados}.

Os dados do dataset estão armazenados na planilha {\tt creditcard.csv}, que é lida e armazenada no {\it dataset} {\tt D}. Todos os atributos são numéricos, exceto a classe das transações, que assumem as {\it strings} {\tt ``0''} ou {\tt ``1''} quando a respectiva transação for legítima ou fraudulenta, respectivamente. A segunda linha converte os elementos do conjunto \{{\tt ``0''}, {\tt ``1''}\} no atributo {\tt Class}, para valores numéricos 0 ou 1. A terceira e quarta linhas separaram as {\it features} das {\it labels}, enquanto a quinta linha treina um modelo de regressão logística utilizando estes dados por 10000 iterações. Em seguida, é utilizado este modelo treinado para realizar predições em um dataset de dados inéditos {\tt x\_test}, que são salvas na planilha {\tt predictions.csv}. Por fim, o modelo (sua estrutura e parâmetros) é salvo e lido do arquivo {\tt model\_name}.

\subsection{Funções predefinidas}

A linguagem fornecerá as seguintes funções predefinidas para o programador:
\\
\\
\texttt{// lê arquivo csv especificado por path e devolve um dataset, com as entradas no arquivo separadas pelo caractere separator
\\
dataset load\_data(string path, \blu{char separator});
\\
\\ 
// guarda o dataset no caminho especificado por path, com as entradas no arquivo separadas pelo caractere separator
\\
void save\_data(dataset d, string path, char separator);
\\
\\
// devolve o mesmo dataset mas com apenas as colunas especificadas
\\ 
dataset cols(dataset d, string[] attributes);
\\
\\
// devolve o mesmo dataset sem as colunas especificadas
\\
dataset remove\_cols(dataset d, string[] attributes);
\\
\\
// devolve o mesmo dataset com exemplos de índice em [begin,begin+num\_samples)
\\
dataset rows(dataset d, int begin, int end);
\\
\\
// devolve o número de linhas (exemplos) do dataset
\\
int num\_rows(dataset d);
\\
\\
// estas funções devolvem modelos baseados nos datasets dados
\\
model perceptron(dataset X, dataset y, int max\_iter);
\\ 
\\
model pocket\_perceptron(dataset X, dataset y, int max\_iter);
\\
\\
model linear\_regression(dataset X, dataset y, \blu{real learning\_rate, int batch\_size, int num\_epochs}, int max\_iter);
\\
\\
model logistic\_regression(dataset X, dataset y, \blu{string feature\_of\_interest, real learning\_rate, int batch\_size, int num\_epochs}, int max\_iter);
\\
\\
// calcula os as saídas y correspondentes a X e devolve y 
\\
dataset predict(\blu{dataset X, model M});
\\
\\
// carrrega modelo armazenado no arquivo especificado pelo caminho path
\\
model load\_model(string path);
\\
\\
// armazena modelo M no caminho path
\\
void save\_model(model M, string path);
}

\color{black}
\section{Levantamento dos Conceitos}

\subsection{Tipos}

Os tipos primitivos da linguagem consistem em inteiro ({\tt int}), valor real de ponto flutuante ({\tt real}), caractere ({\tt char}), booleano ({\tt bool}), além do tipo que representa um modelo preditivo treinado ({\tt model}). O tipo {\tt model} está enquadrado como primitivo devido ao fato de não  haver o interesse atual em permitir acesso aos parâmetros individuais do modelo no nível da linguagem. A palavra reservada {\tt void} poderá ser usada apenas como tipo de retorno de funções, indicando que a função não retorna valor.

Além dos tipos primitivos, existem três tipos compostos na linguagem. Os vetores, que são sequências de qualquer outro tipo, indexadas a partir de zero, o tipo {\tt string}, que representa sequências de caracteres, e o tipo {\tt dataset}, que representa um conjunto de dados. valores do tipo {\tt dataset} são constituídas de uma sequência de exemplos, sendo que os exemplos, por sua vez, são subdivididos em atributos. Será possível acessar intervalos de exemplos dentro do {\it dataset}, e haverá funções predefinidas na linguagem para acessar atributos específicos de todos os exemplos.

Não será possível para o programador criar novos tipos na linguagem.

\subsection{Expressões e Comandos}

A linguagem possuirá as expressões aritméticas e lógicas fundamentais, como qualquer linguagem de propósito geral. \red{Também terá expressões da forma 
{\tt d[2:10]} para especificar {\tt dataset}s menores, no caso apenas com os exemplos de índice entre 2 e 10.}

\blu{Operações do tipo \texttt{d[2:10]}, como no exemplo acima, não existem na versão final da linguagem. Sub-datasets podem ser obtidos através da função predefinida \texttt{rows}.}

Além disto, a linguagem possuirá comandos para atribuição, desvio condicional ({\tt if/else}) e laço de repetição ({\tt while}).

Em detalhes, teremos as expressões e comandos a seguir, onde, quando dizemos que algo tem um certo tipo, estamos dizendo que esse algo é uma expressão cujo valor calculado tem o tal tipo.

Expressões:

\begin{itemize}
\item Literais
	\begin{itemize}
    \item 5 tem tipo {\tt int}
    \item 5.0 tem tipo {\tt real}
    \item 'a' tem tipo {\tt char}
     \item "Caio" tem tipo {\tt string}
    \item true tem tipo {\tt bool}
	\end{itemize}
\item Variáveis
	\begin{itemize}
    \item x tem seu tipo determinado pela declaração visível segundo o escopo, conforme definido na seção seguinte
   	\end{itemize}
\item Agregações
	\begin{itemize}
    \item {\tt\{}$e_1, e_2, \ldots, e_n${\tt\}} tem tipo vetorDe($t$)"
    se cada $e_i$ tem tipo $t$. \blu{Por exemplo, \{"salario", "idade", "credito"\} tem tipo "vetor de {\tt string}'s", pois "salario", "idade" e "credito" têm tipo {\tt string}. Por outro lado, \{12, 5\} têm tipo "vetor de {\tt int}'s". Assim, se o programador declarar \texttt{int[] v = \{12, 5\};}, o teste \texttt{v[1] == 5} resultará no valor \texttt{true}.}
    \end{itemize}
\item Aplicações de funções
	\begin{itemize}
    \item {\tt f($e_1, e_2, \ldots, e_n$)} tem tipo {\tt t} se 
    {\tt f} é uma função de $s_1\times s_2\times\ldots\times s_n$ para
    $t$ e cada $e_i$ tem tipo $s_i$. Note que {\tt f} precisa estar definida
    no programa para que a aplicação seja válida.
    \item Analogamente para as funcões predefinidas.
    \end{itemize}
\item Operações aritméticas
	\begin{itemize}
    \item {\tt $e_1$ + $e_2$} tem tipo {\tt int} se cada $e_i$ tem tipo
    {\tt int} e o ``{\tt +}" está na raiz da {\it parse tree} da expressão
    \item {\tt $e_1$ + $e_2$} tem tipo {\tt real} se cada $e_i$ tem tipo
    {\tt int} ou {\tt real}, pelo menos um dos $e_i$ tem tipo {\tt real}
    e o ``{\tt +}" está na raiz da {\it parse tree} da expressão
    \item Analogamente para {\tt -, *} e {\tt /}
    \end{itemize}
\item Operações lógicas
	\begin{itemize}
    \item {\tt $e_1$ < $e_2$} tem tipo {\tt bool} se cada $e_i$ tem tipo
    {\tt int} ou {\tt float} e o ``{\tt <}" está na raiz da {\it parse tree} da
    expressão
    \item Analogamente para {\tt >, <=, >=, ==, !=}
    \item {\tt $e_1$ \&\& $e_2$} tem tipo {\tt bool} se cada $e_i$ tem tipo
    {\tt bool} e o ``{\tt \&\&}" está na raiz da {\it parse tree} da expressão
    \item Analogamente para {\tt ||}
    \item {\tt !$e_1$} tem tipo {\tt bool} se $e_1$ tem tipo {\tt bool}
    \end{itemize}
\item Operações com vetores
	\begin{itemize}
    \item {\tt v[$e_1$]} tem tipo $t$ se $v$ tem tipo ``vetor de $t$'s"
    e $e_1$ tem tipo {\tt int}
    \end{itemize}
\item \red{Operações com {\it datasets}
	\begin{itemize}
    \item {\tt d[$e_1$:$e_2$]} tem tipo {\tt dataset} se
    $d$ tem tipo {\tt dataset}
    e os $e_i$ têm tipo {\tt int}
    \end{itemize}}
\item Atribuições
	\begin{itemize}
    \item {\tt x = $e_1$} tem tipo $t$ se 
    {\tt x} é uma variável de tipo $t$,
    $e_1$ tem tipo $t$ e
    o ``{\tt =}" está na raiz da {\it parse tree} da expressão
    \item {\tt v[$e_1$] = $e_2$} tem tipo $t$ se 
    $e_1$ tem tipo {\tt int},
    $e_2$ tem tipo $t$, e
    {\tt v} tem tipo ``vetor de {\tt t}'s"
    
    \end{itemize}
\end{itemize}

Comandos:

\begin{itemize}
\item Comando-expressão
	\begin{itemize}
	\item {\tt $e_1$;} é comando se $e_1$ é expressão
    \item {\tt ;} é comando
	\end{itemize}
\item Desvio condicional
	\begin{itemize}
	\item {\tt if($e_1$) $c_1$} é comando se $e_1$ tem tipo {\tt bool} e 
    $c_1$ é comando composto
    \item {\tt if($e_1$) $c_1$ else $c_2$} é comando se 
    $e_1$ tem tipo {\tt bool} e
    $c_1$ e $c_2$ são comandos compostos
	\end{itemize}
\item Laço
	\begin{itemize}
    \item {\tt while($e_1$) $c_1$} é comando se
    $e_1$ tem tipo {\tt bool} e
    $c_1$ é comando composto
    \end{itemize}
\item Comando composto
	\begin{itemize}
    \item {\tt\{$c_1$ $c_2$ $\ldots$ $c_n$\}} é comando se os $c_i$ forem comandos
    ou declarações (onde $n\geq \blu{1}$)
    \end{itemize}
\end{itemize}

\subsection{Vinculação e Escopo}

A vinculação das variáveis será feita de modo explícito. Toda variável deve ser declarada no código precedida do um identificador do tipo, como por exemplo:

\begin{verbatim}
dataset D = read_data(`data.csv');
\end{verbatim}

Quanto ao tempo de vinculação, as variáveis serão vinculadas de forma estática. Ou seja, antes do tempo de execução do programa, permancendo inalteradas durante a execução.

Como não é permitida a criação de novos tipos na linguagem, a vinculação de tipos a suas palavras reservadas já está embutida no processador da linguagem, e portanto, também ocorre de forma estática.

O escopo será estático: uma variável estará visível num bloco se ela for local àquele bloco ou a algum bloco que o contenha.\footnote{Note que a contenção é uma relação transitiva: se o bloco {\it b1} contém o bloco {\it b2}, e o bloco {\it b2} contém o bloco {\it b3}, então {\it b1} contém {\it b3}.} A exceção ocorre quando duas variáveis de mesmo nome estão no bloco ou em algum bloco que o contém: neste caso, a mais interior será visível e a mais exterior, não.

\subsection{Sistema de Tipos}
O sistema de tipos será monomórfico: variáveis, constantes e parâmetros de funções serão sempre declaradas com um único tipo. Haverá sobrecarga dos operadores de soma, subtração, multiplicação e divisão,  que poderão operar sobre {\tt int}s e {\tt real}s. A sobrecarga será independente de contexto: olhando-se apenas para os operandos, se deduzirá o operador a ser utilizado.


\subsection{Verificação de Tipos}
Na verificação de tipos, será feita a coerção de {\tt int} para {\tt real}, tanto em expressões aritméticas como em declarações da forma {\tt real x = 1;}. Como o programador não definirá seus próprios tipos, a equivalência de tipos se reduzirá à equivalência nominal: dois tipos são equivalentes se, e somente se, são iguais. Finalmente, sendo os tipos explicitados no código, não será feita qualquer inferência de tipos.

\subsection{Avaliação de Expressões}
Parênteses definirão parcialmente a ordem de avaliação de expressões. Caso ainda haja ambiguidade, a precedência de operadores a seguir será utilizada:

\begin{enumerate}
\item {\tt !}
\item {\tt *, /}
\item {\tt +, -}
\item {\tt <, <=, >, >=}
\item {\tt ==, !=}
\item {\tt \&\&}
\item {\tt ||}
\item {\tt =}

No caso de operadores com empate de precedência, a execução das operações será realizada da esquerda para a direita, com exceção da atribuição ('=') e da negação unária ('!'), que terão associatividade à direita.

\end{enumerate}

\subsection{Abstrações}
A linguagem não terá abstrações de tipo, já que o programador não poderá definir seus próprios tipos.

Haverá abstração de funções e procedimentos sob o rótulo ``função", assim como em {\it C}. Isto é, procedimentos serão declarados como funções que retornam {\tt void}.

A passagem de parãmetros será exclusivamente pelo mecanismo de cópia. Há aqui uma sutileza: em {\it C}, vetores decaem para ponteiros ao serem passados como argumento e, portanto, são passados por referência; em {\it CML} não há ponteiros e os vetores são copiados.

Parametrização de tipos não será possível na linguagem e os argumentos das funções serão avaliados {\it a priori}.

\end{document}